%-----------------------------------------------------------------------------%
%	Licença
%-----------------------------------------------------------------------------%
%This template has been downloaded from:
% http://www.LaTeXTemplates.com
%
% Original author:
% Mathias Legrand (legrand.mathias@gmail.com)
%
% License:
% CC BY-NC-SA 3.0 (http://creativecommons.org/licenses/by-nc-sa/3.0/)

%-----------------------------------------------------------------------------%
%	Pacotes e outras configurações
%-----------------------------------------------------------------------------%
\documentclass[a4paper,10pt]{SelfArx}
\usepackage[brazilian]{babel}
\usepackage[utf8]{inputenc}
\usepackage{epsf,epsfig} % include graphics [pdf, png etc]
\usepackage{setspace}
\usepackage{amsmath} % allow the use of mathematical formulas
\usepackage{textcomp} % which provide many text symbols
\usepackage{natbib} %bibtex flexible bibliography support.
\usepackage[version=3]{mhchem} %chemistry package
\usepackage{float}
\restylefloat{table}
\usepackage{titlesec}
%\titlespacing{\section}

%-----------------------------------------------------------------------------%
%	Adicionar a Watermark
%-----------------------------------------------------------------------------%
% to insert draft-like watermark
% \usepackage{draftwatermark} %Put a grey textual watermark on document pages.
% \SetWatermarkAngle{45}
% \SetWatermarkLightness{0.8}
% \SetWatermarkFontSize{5cm}
% \SetWatermarkScale{5}
% \SetWatermarkText{XXX} % XXX = text that I want to see as watermark

%-----------------------------------------------------------------------------%
%	Colunas
%-----------------------------------------------------------------------------%
\setlength{\columnsep}{0.55cm} % Distance between the two columns of text
\setlength{\fboxrule}{0.75pt} % Width of the border around the abstract

%-----------------------------------------------------------------------------%
%	Cores
%-----------------------------------------------------------------------------%
\definecolor{color1}{RGB}{0,0,90} % Color of the article title and sections
\definecolor{color2}{RGB}{0,20,20} % Color of the boxes behind the abstract and headings

%-----------------------------------------------------------------------------%
%	Inforamações do Artigo 
%-----------------------------------------------------------------------------%
%\pdfinfo{%
 % /Title    (Guia para as aulas práticas)
  %/Author   (Ju Leonel)
  %/Subject  (Oceanografia Química)
  %}

\JournalInfo{Guia de Análises Químicas - Gravimetria} % Journal information
\Archive{Material Particulado em Suspensão} % Additional notes (e.g. copyright, DOI, review/research article)
\PaperTitle{Determinação de Material Particulado em Suspensão } % Article title
\Authors{Disciplina: de Geoquímica - Curso: Oceanografia - 29 de setembro de 2014} % Authors
\affiliation{~} % Corresponding author
% Keywords - if you don't want any simply remove all the text between the curly brackets
\Keywords{}
\newcommand{\keywordname}{~} % Defines the keywords heading name


%-----------------------------------------------------------------------------%
%	ABSTRACT
%-----------------------------------------------------------------------------%
%change ABSTRACT to something else
\addto{\captionsbrazilian}{\renewcommand{\abstractname}{Objetivo da Prática}}
\Abstract{O objetivo dessa atividade é usar o método gravimétrico para determinar a quantidade de material particulado na amostras de água coletada na disciplina de Geoquímica. A água filtrada será armazenada para futuras análises de metais e fosfato e a os filtros serão armazenados para análise de metais e carbono orgânico.}
%-----------------------------------------------------------------------------%
\begin{document}
\flushbottom % Makes all text pages the same height
\maketitle % Print the title and abstract box
\renewcommand{\contentsname}{Conteúdo}
\tableofcontents % Print the contents section
\thispagestyle{empty} % Removes page numbering from the first page

%-----------------------------------------------------------------------------%
%	ARTICLE CONTENTS
%-----------------------------------------------------------------------------%
\section*{Método de Análise: gravimétrico} % The \section*{} command stops section numbering
\addcontentsline{toc}{section}{Método de Análise: Gravimetria} % Adds this section to the table of contents
O método de análise apresentado aqui baseia-se no descrito por Strickland e Parsons (1972).
%-----------------------------------------------------------------------------%

\section{Fundamento Analítico}

A separação entre material em particulado e material dissolvido é operacional: partículas com diâmetro maiores que 0,45 µm são consideradas em suspensão e partículas menores dissolvidas. Portanto, a separação e quantificação do material particulado em suspensão e realizado pela filtração de um volume conhecido de água.
%-----------------------------------------------------------------------------%

\section{Procedimentos Analíticos}
% [noitemsep] removes whitespace between the items for a compact look
\subsection{Preparação dos Filtros}
\begin{enumerate}[noitemsep]
\item Calcinar os filtros a 450°C por 4 horas
\item Armazenar em dessecador até estarem frios
\item Pesar o filtro em balança analítica manuseando com a pinça
\item Anotar o peso de cada filtro (P1)
\end{enumerate}

\subsection{Filtração das amostras}
\begin{enumerate}[noitemsep]
\item Montar o equipamento de filtragem e posicionar um filtro sobre o suporte poroso
\item Agitar vagarosamente a amostra para homogeneizar
\item Medir o volume a ser filtrado na proveta (V)
\item Ligar a bomba e começar a derramar lentamente a amostra sobre o béquer do equipamento
\item Após filtrar a amostra retirar a amostra do kitassato e armazenar adequadamente para futuras análises
\item Reposicionar o kitassato e enxaguar o filtro com aproximadamente 5mL (para amostras estuarinas) de água destilada para remoção dos cloretos do filtro – quanto mais salina for a amostra mais volume de água deve ser utilizado para a lavagem.
\item Após a filtragem retirar cuidadosamente o filtro com a bomba desligada. Colocar o filtro sobre uma placa de Petry já etiquetada ou sobre um envelope de papel
\item Secar em estufa a 60°C
\item Após retirar da estufa colocar no dessecador até esfriar
\item Pesar em balança analítica (P2)

\end{enumerate}

\subsection{Branco}
Para cada grupo de amostras deve-se fazer um branco do método. Nesse procedimento o filtro é calcinado e pesado (B1), mas não é colocado no equipamento. Esse filtro também e lavado com água destilada, seco em estufa e repesado (B2). A diferença entre B2 e B1 e o valor Br a ser usado no calculo do material e suspensão das amostras.	
%-----------------------------------------------------------------------------%

\section{Cálculos}

\indent 
Grasshoff e Wenk
\textbf {Branco}
\begin{center}
$\ Br = (P2_{Br}-P1_{Br})$
\end{center}
\vspace{0.15cm}

Onde:

$P1_{Br}$ = peso inicial do filtro (g)

$P2_{Br}$ = peso final do filtro (g)

\vspace{0.20cm}

\textbf {Material Particulado em Suspensão}

\vspace{0.15cm}

\begin{center}
$\{[(P2_{Am}-P1_{Am}) - Br] * 10^6\}/V$
\end{center}
\vspace{0.15cm}
 
Onde:

$P1_{Am}$ = peso inicial do filtro (g)

$P2_{Am}$ = peso final do filtro (g)

Br = valor da prova em branco (g)

$10^6$ = fator de conversão de unidade de g $mL^{-1}$ para $mL^{-1}$

V = volume de amostra filtrada (L)

\vspace{0.2cm}

%-----------------------------------------------------------------------------%
%	Lista de Referencias
%-----------------------------------------------------------------------------%
%\phantomsection
\section{Referências}
Strickland, J. D. H. e Parsons, T. R. (1972). A practical handbook of seawater analysis. $2^o$ Ed. - Ottawa: Fisheries Research Board of Canada, Bulletin 167, 311p.
\bibliographystyle{unsrt}
% \bibliography{sample}

%-----------------------------------------------------------------------------%
%	Tabelas
%-----------------------------------------------------------------------------%
\section{Tabelas - Resultados}

\begin{table}[H]
\centering
\begin{tabular}{|l|c|r|}
\hline
\hline
Peso 1 (g) & Peso 2 (g) & Volume (mL) \\
\hline
\hline
&  &   \\
\hline
&  &   \\
\hline
&  &   \\
\hline
&  &   \\
\hline
&  &   \\
\hline
&  &   \\
\hline
&  &   \\
\hline
&  &   \\
\hline
&  &   \\
\hline
&  &   \\
\hline
&  &   \\
\hline
&  &   \\
\hline
&  &   \\
\hline
&  &   \\  [1ex] % [1ex] adds vertical space
\hline

\hline
\end{tabular}
\caption{Tabela para Calculo de MPS}
\label{ex:Tabela 1}
\end{table}
\end{document}



