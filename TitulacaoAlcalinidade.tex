%-----------------------------------------------------------------------------%
%	Licença
%-----------------------------------------------------------------------------%
%This template has been downloaded from:
% http://www.LaTeXTemplates.com
%
% Original author:
% Mathias Legrand (legrand.mathias@gmail.com)
%
% License:
% CC BY-NC-SA 3.0 (http://creativecommons.org/licenses/by-nc-sa/3.0/)

%-----------------------------------------------------------------------------%
%	Pacotes e outras configurações
%-----------------------------------------------------------------------------%
\documentclass[a4paper,10pt]{SelfArx}
\usepackage[brazilian]{babel}
\usepackage[utf8]{inputenc}
\usepackage{epsf,epsfig} % include graphics [pdf, png etc]
\usepackage{setspace}
\usepackage{amsmath} % allow the use of mathematical formulas
\usepackage{textcomp} % which provide many text symbols
\usepackage{natbib} %bibtex flexible bibliography support.
\usepackage[version=3]{mhchem} %chemistry package
\usepackage{float}
\restylefloat{table}
\usepackage{titlesec}
%\titlespacing{\section}

%-----------------------------------------------------------------------------%
%	Adicionar a Watermark
%-----------------------------------------------------------------------------%
% to insert draft-like watermark
% \usepackage{draftwatermark} %Put a grey textual watermark on document pages.
% \SetWatermarkAngle{45}
% \SetWatermarkLightness{0.8}
% \SetWatermarkFontSize{5cm}
% \SetWatermarkScale{5}
% \SetWatermarkText{XXX} % XXX = text that I want to see as watermark

%-----------------------------------------------------------------------------%
%	Colunas
%-----------------------------------------------------------------------------%
\setlength{\columnsep}{0.55cm} % Distance between the two columns of text
\setlength{\fboxrule}{0.75pt} % Width of the border around the abstract

%-----------------------------------------------------------------------------%
%	Cores
%-----------------------------------------------------------------------------%
\definecolor{color1}{RGB}{0,0,90} % Color of the article title and sections
\definecolor{color2}{RGB}{0,20,20} % Color of the boxes behind the abstract and headings

%-----------------------------------------------------------------------------%
%	Inforamações do Artigo 
%-----------------------------------------------------------------------------%
%\pdfinfo{%
 % /Title    (Guia para as aulas práticas)
  %/Author   (Ju Leonel)
  %/Subject  (Oceanografia Química)
  %}

\JournalInfo{Guia de Oceanografia Química - Titulação} % Journal information
\Archive{Determinação de Alcalinidade} % Additional notes (e.g. copyright, DOI, review/research article)
\PaperTitle{Determinação de Alcalinidade da Água do Mar } % Article title
\Authors{Aula Prática de Oceanografia Química - 19 e 20 de novembro de 2014} % Authors
\affiliation{~} % Corresponding author
% Keywords - if you don't want any simply remove all the text between the curly brackets
\Keywords{}
\newcommand{\keywordname}{~} % Defines the keywords heading name


%-----------------------------------------------------------------------------%
%	ABSTRACT
%-----------------------------------------------------------------------------%
%change ABSTRACT to something else
\addto{\captionsbrazilian}{\renewcommand{\abstractname}{Objetivo da Prática}}
\Abstract{O objetivo dessa atividade é demonstrar o método de determinação de alcalinidade na água do mar por titulação.}

%-----------------------------------------------------------------------------%
\begin{document}
\flushbottom % Makes all text pages the same height
\maketitle % Print the title and abstract box
\renewcommand{\contentsname}{Conteúdo}
\tableofcontents % Print the contents section
\thispagestyle{empty} % Removes page numbering from the first page

%-----------------------------------------------------------------------------%
%	ARTICLE CONTENTS
%-----------------------------------------------------------------------------%
\section*{Método de Análise: Mohr-Nudsen} % The \section*{} command stops section numbering
\addcontentsline{toc}{section}{Método de Análise: Mohr-Nudsen} % Adds this section to the table of contents
O método de análise apresentado aqui baseia-se no descrito pela American Public Health Association (APHA, 1998).

%-----------------------------------------------------------------------------%
\section{Fundamento Analítico}

Uma das definições de alcalinidade é a sua capacidade quantitativa de neutralizar um ácido forte, até um pH desejado. Portanto, é a soma de todos os componentes com caráter básico tituláveis, e o valor medido pode variar significativamente com o ponto final da daterminação. 

A alcalinidade da água marinha baseia-se na sua titulação com ácido o qual neutraliza ânions oriundos de ácidos fracos, tais como bicarbonato, carbonato e borato. A água marinha tem alcalinidade total constante de $2,34\times 10^{-3}$ mol de carbonato por Kg e a concentração média de íon bicarbonato é de 0,14 mg $Kg^{-1}$. Ânions fosfato, arseniato e silicato existem em quantidades traços na água marinha e portanto suas concentrações são desprezíveis 

A alcalinidade total da água do mar refere-se à soma de ânions com maior representatividade: bicarbonatos (89,8\%), carbonatos (6,7\%) e boratos (2,9\%).

\vspace{0.15cm}
$A_T$ = [\ce{HCO3-}] + 2[\ce{CO3^2-}] + [\ce {B(OH)4-}] + [\ce{OH-}] - [\ce{H+}]
\vspace{0.15cm}

A alcalinidade carbonática é representada pelos íons bicarbonato e carbonato.

\vspace{0.15cm}
$A_C$ = [\ce{HCO3-}] + 2[\ce{CO3^2-}] 
\vspace{0.15cm}

Dessa forma, o presente método se baseia na titulação da água do mar com ácido forte. O indicador fenolftaleína permite a medida da “alcalinidade carbonática” e o indicadr alaranjado de metila permite medir a “alcalinidade total”.

%-----------------------------------------------------------------------------%
\section{Amostragem e Estocagem}
As amostras para determinação de AT devem ser coletadas em frascos de vidro de borosilicato de cor âmbar, com capacidade de 300 mL, com tampa de vidro esmerilhada e de boca estreita.

A partir da garrafa a coleta é feita logo após a amostragem de pH usando-se o mesmo dispositivo, ou seja, um tubo de plástico ou vidro com ponta afilada adaptada à saída da garrafa hidrográfica, que possa ser mergulhado no fundo do frasco.

A amostra deve ser retirada lentamente, deixando-se extravasar um volume da mesma antes de tampar. A contaminação atmosférica deve ser rigorosamente evitada, não deve haver bolhas de ar sob a tampa e o frasco deve ser fechado hermeticamente.
%-----------------------------------------------------------------------------%
\section{Procedimentos Analíticos}

\subsection{Reagentes}

% [noitemsep] removes whitespace between the items for a compact look
\begin{enumerate}[noitemsep]
\item Ácido clorídrico 0,02M 
\item Indicador fenolftaleína -  preparar uma solução a 1\% de massa/volume em álcool etílico a 96 GL
\item Indicador verde de bromocresol - dissolver 0,1 g de sal sódico de verde bromocresol em 100 mL de água destialda (pH 4,5)
\item Indicador alaranjado de metila - preparar uma solução a 1\% em massa/volume em álcool etílico a 96GL.
\end{enumerate}

\subsection{Titulação das Amostras}
% [noitemsep] removes whitespace between the items for a compact look
\begin{enumerate}[noitemsep]
\item Adicionar em 100 mL de amostra de 5 a 10 gotas do indicador fenolftaleína 
\item Titular com uma solução padrão de 0,02 M de ácido clorídrico até mudança de cor da solução para incolor.
\ item Repetir a titulação com 5 a 10 gotas de indicador alaranjado de metila.
\end{enumerate}

%-----------------------------------------------------------------------------%
\section{Cálculos}

\indent 
\textbf {Cálculo da alcalinidade parcial (carbonática)}

\begin{center}
$[A_{C}] = (V1_{HCl} * [HCl])/V_{am}$
\end{center}

Onde:

$V1_{HCl}$ = volume de HCl usado na titulação con fenolftaleína

$[HCl]$ = concentração do HCl

$V_{am}$ = volume de amostra utilizada


\vspace{0.15cm}
\noindent 
\textbf {Cálculo da alcalinidade total}

\begin{center}
$[A_{T}] = (V2_{HCl} * [HCl])/V_{am}$
\end{center}

Onde:

$V2_{HCl}$ = volume de HCl usado na titulação com indicador alaranjado de metila

$[HCl]$ = concentração do HCl usado na titulação

$V_{am}$ = volume de amostra utilizada




 
%-----------------------------------------------------------------------------%
%	Lista de Referencias
%-----------------------------------------------------------------------------%
%\phantomsection
\section{Referências}
American Public Health Association (APHA)(1998). Standard methods for the examination of water and wastewater. Washington. 1193p.

\bibliographystyle{unsrt}
% \bibliography{sample}

\end{document}



