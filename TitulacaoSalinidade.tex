%-----------------------------------------------------------------------------%
%	Licença
%-----------------------------------------------------------------------------%
%This template has been downloaded from:
% http://www.LaTeXTemplates.com
%
% Original author:
% Mathias Legrand (legrand.mathias@gmail.com)
%
% License:
% CC BY-NC-SA 3.0 (http://creativecommons.org/licenses/by-nc-sa/3.0/)

%-----------------------------------------------------------------------------%
%	Pacotes e outras configurações
%-----------------------------------------------------------------------------%
\documentclass[a4paper,10pt]{SelfArx}
\usepackage[brazilian]{babel}
\usepackage[utf8]{inputenc}
\usepackage{epsf,epsfig} % include graphics [pdf, png etc]
\usepackage{setspace}
\usepackage{amsmath} % allow the use of mathematical formulas
\usepackage{textcomp} % which provide many text symbols
\usepackage{natbib} %bibtex flexible bibliography support.
\usepackage[version=3]{mhchem} %chemistry package
\usepackage{float}
\restylefloat{table}
\usepackage{titlesec}
%\titlespacing{\section}

%-----------------------------------------------------------------------------%
%	Adicionar a Watermark
%-----------------------------------------------------------------------------%
% to insert draft-like watermark
% \usepackage{draftwatermark} %Put a grey textual watermark on document pages.
% \SetWatermarkAngle{45}
% \SetWatermarkLightness{0.8}
% \SetWatermarkFontSize{5cm}
% \SetWatermarkScale{5}
% \SetWatermarkText{XXX} % XXX = text that I want to see as watermark

%-----------------------------------------------------------------------------%
%	Colunas
%-----------------------------------------------------------------------------%
\setlength{\columnsep}{0.55cm} % Distance between the two columns of text
\setlength{\fboxrule}{0.75pt} % Width of the border around the abstract

%-----------------------------------------------------------------------------%
%	Cores
%-----------------------------------------------------------------------------%
\definecolor{color1}{RGB}{0,0,90} % Color of the article title and sections
\definecolor{color2}{RGB}{0,20,20} % Color of the boxes behind the abstract and headings

%-----------------------------------------------------------------------------%
%	Inforamações do Artigo 
%-----------------------------------------------------------------------------%
%\pdfinfo{%
 % /Title    (Guia para as aulas práticas)
  %/Author   (Ju Leonel)
  %/Subject  (Oceanografia Química)
  %}

\JournalInfo{Guia de Oceanografia Química - Titulação} % Journal information
\Archive{Determinação de Clorinidade} % Additional notes (e.g. copyright, DOI, review/research article)
\PaperTitle{Determinação de Clorinidade da Água do Mar } % Article title
\Authors{Aula de Oceanografia Química - 22 e 23 de outubro de 2014} % Authors
\affiliation{~} % Corresponding author
% Keywords - if you don't want any simply remove all the text between the curly brackets
\Keywords{}
\newcommand{\keywordname}{~} % Defines the keywords heading name


%-----------------------------------------------------------------------------%
%	ABSTRACT
%-----------------------------------------------------------------------------%
%change ABSTRACT to something else
\addto{\captionsbrazilian}{\renewcommand{\abstractname}{Objetivo da Prática}}
\Abstract{O objetivo dessa atividade é demonstrar o método de determinação de clorinidade na água do mar por titulação de precipitação.}

%-----------------------------------------------------------------------------%
\begin{document}
\flushbottom % Makes all text pages the same height
\maketitle % Print the title and abstract box
\renewcommand{\contentsname}{Conteúdo}
\tableofcontents % Print the contents section
\thispagestyle{empty} % Removes page numbering from the first page

%-----------------------------------------------------------------------------%
%	ARTICLE CONTENTS
%-----------------------------------------------------------------------------%
\section*{Método de Análise: Titulação} % The \section*{} command stops section numbering
\addcontentsline{toc}{section}{Método de Análise: Titulação} % Adds this section to the table of contents
O método de análise apresentado aqui baseia-se no descrito por Mohr-Nudsen e modificado por Grasshoff e Wenk (1972).
%-----------------------------------------------------------------------------%
\section{Fundamento Analítico}

O presente método baseia-se na titulação dos íons haletos \ce{(Cl^- , Br^- , I^-)}
presentes na água do mar, com solução pa\-dro\-ni\-za\-da de nitrato de prata \ce{(AgNO3)},
usando solução de cromato de potássio \ce{(K2CrO4)} como indicador.
Durante a titulação ocorre a precipitação dos seguintes sais:
\vspace{0.15cm}

\ce {Cl^-_{(aq)}  + Ag+_{(aq)} ->  AgCl_{(s)} v } (branco)

K$_{ps}$ = 1,55 $\times$ 10$^{-10}$
\vspace{0.15cm}

\ce {Br^-_{(aq)}  + Ag+_{(aq)} ->  AgBr_{(s)} v} (amarelo)   

K$_{ps}$ = 7,76 $\times$ 10$^{-13}$
\vspace{0.15cm}

\ce {I^-_{(aq)}  + Ag+_{(aq)} ->  AgI_{(s)} v} (laranja)

K$_{ps}$ = 1,51 $\times$ 10$^{-16}$
\vspace{0.15cm}

O ponto final da titulação é indicado pela formação do cromato de prata:
\vspace{0.15cm}

\ce {CrO4^{2-}_{(aq)} + Ag+_{(aq)} ->  Ag2CrO4_{(s)} v} (marrom avermelhado) 

Kps = 8,91 $\times$ 10$^{-12}$
\vspace{0.15cm}

Se a concentração do íon cromato \ce{(CrO4^{2-})} for mantida em aproximadamente
5,0 $\times$ 10$^{-3}$ mol L$^{-1}$ durante a titulação, não ocorrerá a precipitação de \ce{Ag2CrO4} até que a concentração de íon prata \ce{(Ag+)} seja igual a 4,2 $\times$ 10$^{-5}$ mol L$^{-1}$.

\textbf{Faixa de Concentração:} o método descrito atende praticamente todo o intervalo de clorinidade em águas marinhas naturais, ou seja, de 0 a 22 $\times$ 10$^{-3}$.

\textbf{Precisão:} o desvio padrão é igual a ±0,01 $\times$ 10$^{-3}$.
%----------------------------------------------------------19,375-------------------%
\section{Amostragem e Estocagem}
As amostras devem ser coletadas em frascos de vidro de 250 mL com tampas para evitar contaminação e evaporação. Antes da amostragem, lavar o frasco com a própria amostra três vezes e, após a coleta, secar a borda do frasco com papel absorvente para evitar o acúmulo de sal. Armazenar as amostras e a água do mar padrão no laboratório onde a determinação será realizada para que alcancem a temperatura ambiente.

%-----------------------------------------------------------------------------%
\section{Procedimentos Analíticos}

\subsection{Reagentes}

% [noitemsep] removes whitespace between the items for a compact look
\begin{enumerate}[noitemsep]
\item \textbf{Solução de cromato de potássio \ce{(K2CrO4)} 8\%} - dissolver 8 g de \ce{(K2CrO4)} em 100 mL de água destilada. Guardar em frasco conta-gotas.
\item \textbf{Solução de nitrato de prata \ce{(AgNO3)} 0,05M} – dissolver 8 g de \ce{(AgNO3)} em 1000 mL de água destilada. Conservar em frasco de vidro âmbar com tampa de borracha ou vidro.
\item \textbf{Água do mar padrão} - preparado segundo Kester et al. (1967) (Tabela 1). 
\end{enumerate}


\subsection{Padronização do nitrato de prata}
% [noitemsep] removes whitespace between the items for a compact look
\begin{enumerate}[noitemsep]
\item Agitar a solução de \ce{(AgNO3)} vigorosamente antes de usar;
\item Pipetar 10 mL da água do mar padrão em um erlenmeyer de 150 mL.  Antes de coletar o volume definitivo, esta pipeta deve ser cuidadosamente lavada;
\item Adicionar 25 mL de água destilada e 6 gotas da solução de cromato de potássio;
\item Deixar a mistura sob agitação (mas sem espirrar a\-mo\-stra);
\item Adicionar a solução de nitrato de prata até que haja um primeiro sinal de mudança de cor;
\item Interromper a titulação até que a cor volte ao amarelo;
\item Lavar as gotas da parede do béquer com água destilada;
\item Recomeçar a adição de solução de nitrato de prata, gota a gota, cuidadosamente até que a solução se torne marrom avermelhada. Essa cor deverá persistir por 30 segundos, sob agitação;
\item Anotar o volume de solução consumido;
\item Repetir a padronização mais duas vezes. Os resultados não podem diferir mais do que 0,01 mL. A média aritmética é usada no cálculo da concentração da so\-lu\-ção;
\item Transferir o conteúdo do béquer para um depósito apropriado e lavá-lo juntamente com o bastão magnético.
\end{enumerate}

\subsection{Titulação das Amostras}
%[noitemsep] removes whitespace between the items for a compact look
\begin{enumerate}[noitemsep]
 \item Proceder da mesma maneira que a padronização de \ce{(AgNO3)}, tomando-se os mesmos cuidados e precauções;
 \item Anotar o volume de solução de \ce{(AgNO3)} consumido e repetir a titulação mais uma vez.
\end{enumerate}

%-----------------------------------------------------------------------------%
\section{Cálculos}

\textbf {Cálculo do fator de calibração “F”}

\begin{center}
$F = N/Cm$
\end{center}

Onde:

N = clorinidade nominal da água do mar padrão (21.105)

Cm = volume médio da padronização do \ce{(AgNO3)}

\vspace{0.20cm}
\noindent 
\textbf {Cálculo da clorinidade}

\begin{center}
$[Cl] = Vm \times F + k$
\end {center}
    
Onde:

Vm = média dos volumes de titulação com \ce{(AgNO3)}

k =valor de correção (Tabela 2)
 
\vspace{0.20cm} 
\noindent 
\textbf {Cálculo de salinidade}

\begin{center}
$S = Cl \times 1,80655$
\end{center}

%-----------------------------------------------------------------------------%
%	Lista de Referencias
%-----------------------------------------------------------------------------%
%\phantomsection
\section{Referências}
Grasshoff, K. \& Wenck, A. (1972). A Modern Version of the Mohr-Knudsen Titration for the Chlorinity of Sea Water.

Kester, D. R., Duedall, I. W., Connors, D. N. \& Pytkowicz, R. M. (1967). Preparation of Artificial Seawater. Limnology \& Oceanography 12, 176—179.

\bibliographystyle{unsrt}
% \bibliography{sample}

%-----------------------------------------------------------------------------%
%	Tabelas
%-----------------------------------------------------------------------------%
\vfill
\section{Tabelas}

\begin{table}[H]
\centering
\begin{tabular}{|l|c|r|}
\hline
\hline
Sais anidros & \\
\hline
\hline
Sal & g kg$^{-1}$ de solução \\
\hline
\ce{NaCl} & 23,926  \\
\ce {Na2SO4} & 4,008 \\
\ce {KCl} & 0,677 \\
\ce {NaHCO3} & 0,19 \\
\ce {KBr} & 0,098 \\
\ce {H3BO3} & 0,026 \\
\ce {NaF} & 0,003 \\
\hline
\hline
Sais hidratados & \\
\hline
\hline
Sal & moles kg$^{-1}$ de solução \\
\hline
\ce {MgCl2 . 6H2O} & 0,05327 \\
\ce {CaCl2 . 2H2O} & 0,01033 \\
\ce {SrCl2 . 6H2O} & 0,00009 \\
\hline
\hline
1 kg água destilada \\
\hline
\hline


\end{tabular}
\caption{Água do Mar Artificial}
\label{ex:Tabela 1}
\end{table}




\begin{table}[H]
\centering
\begin{tabular}{|l|c|r|}
\hline
\hline
Volume & Volume & K \\
\hline
\hline
& 23,18 -- 22,89  & -0,11  \\
& 22,89 -- 22,59 & -0,10  \\
& 22,59 -- 22,29 & -0,09  \\
& 22,29 -- 21,98 & -0,08  \\
& 21,98 -- 21,66 & -0,07  \\
& 21,66 -- 21,35 & -0,06  \\
& 21,35 -- 21,01 & -0,05  \\
& 21,01 -- 20,67 & -0.04  \\
& 20,67 -- 20,31 & -0.03  \\
& 20,31 -- 19,94 & -0,02  \\
& 19,94 -- 19,57 & -0,01  \\
0,00 -- 0,18 & 19,57 -- 19,17 & 0,00  \\
0,18 -- 0,58 & 19,17 -- 18,77 & 0,01  \\
0,58 -- 0,99 & 18,77 -- 18,34 & 0,02  \\
0,99 -- 1,42 & 18,34 -- 17,89 & 0,03  \\
1,42 -- 1,88 & 17,89 -- 17,41 & 0,04  \\
1,88 -- 2,36 & 17,41 -- 16,91 & 0,05  \\
2,36 -- 2,88 & 16,91 -- 16,37 & 0,06  \\
2,88 -- 3,45 & 16,37 -- 15,78 & 0,07  \\
3,45 -- 4,08 & 15,78 -- 15,13 & 0,08  \\
4,08 -- 4,79 & 15,13 -- 14,40 & 0,09  \\
4,79 -- 5,64 & 14,40 -- 13,32 & 0,10  \\
5,64 -- 6,74 & 13,32 -- 12,39 & 0,11  \\
6,74 -- 8,82 & 12,39 -- 10,29 & 0,12 \\
8,82 -- 10.29 & 10,29 -- 8,82 & 0,13  \\ [1ex] % [1ex] adds vertical space
\hline
\end{tabular}
\caption{Valores de k}
\label{ex:Tabela 1}
\end{table}



\end{document}



