%-----------------------------------------------------------------------------%
%	Licença
%-----------------------------------------------------------------------------%
%This template has been downloaded from:
% http://www.LaTeXTemplates.com
%
% Original author:
% Mathias Legrand (legrand.mathias@gmail.com)
%
% License:
% CC BY-NC-SA 3.0 (http://creativecommons.org/licenses/by-nc-sa/3.0/)

%-----------------------------------------------------------------------------%
%	Pacotes e outras configurações
%-----------------------------------------------------------------------------%
\documentclass[a4paper,10pt]{SelfArx}
\usepackage[brazilian]{babel}
\usepackage[utf8]{inputenc}
\usepackage{epsf,epsfig} % include graphics [pdf, png etc]
\usepackage{setspace}
\usepackage{amsmath} % allow the use of mathematical formulas
\usepackage{textcomp} % which provide many text symbols
\usepackage{natbib} %bibtex flexible bibliography support.
\usepackage[version=3]{mhchem} %chemistry package
\usepackage{float}
\restylefloat{table}
\usepackage{titlesec}
%\titlespacing{\section}

%-----------------------------------------------------------------------------%
%	Adicionar a Watermark
%-----------------------------------------------------------------------------%
% to insert draft-like watermark
% \usepackage{draftwatermark} %Put a grey textual watermark on document pages.
% \SetWatermarkAngle{45}
% \SetWatermarkLightness{0.8}
% \SetWatermarkFontSize{5cm}
% \SetWatermarkScale{5}
% \SetWatermarkText{XXX} % XXX = text that I want to see as watermark

%-----------------------------------------------------------------------------%
%	Colunas
%-----------------------------------------------------------------------------%
\setlength{\columnsep}{0.55cm} % Distance between the two columns of text
\setlength{\fboxrule}{0.75pt} % Width of the border around the abstract

%-----------------------------------------------------------------------------%
%	Cores
%-----------------------------------------------------------------------------%
\definecolor{color1}{RGB}{0,0,90} % Color of the article title and sections
\definecolor{color2}{RGB}{0,20,20} % Color of the boxes behind the abstract and headings

%-----------------------------------------------------------------------------%
%	Inforamações do Artigo 
%-----------------------------------------------------------------------------%
%\pdfinfo{%
 % /Title    (Guia para as aulas práticas)
  %/Author   (Ju Leonel)
  %/Subject  (Oceanografia Química)
  %}

\JournalInfo{Guia de Oceanografia Química - Titulação} % Journal information
\Archive{Determinação de Oxigênio Dissolvido} % Additional notes (e.g. copyright, DOI, review/research article)
\PaperTitle{Determinação de Oxigênio Dissolvido na Água do Mar } % Article title
\Authors{Aula Prática de Oceanografia Química - 05 e 06 de novembro de 2014} % Authors
\affiliation{~} % Corresponding author
% Keywords - if you don't want any simply remove all the text between the curly brackets
\Keywords{}
\newcommand{\keywordname}{~} % Defines the keywords heading name


%-----------------------------------------------------------------------------%
%	ABSTRACT
%-----------------------------------------------------------------------------%
%change ABSTRACT to something else
\addto{\captionsbrazilian}{\renewcommand{\abstractname}{Objetivo da Prática}}
\Abstract{O objetivo dessa atividade é demonstrar o método de determinação de oxigênio dissolvida na água do mar por titulação.}

%-----------------------------------------------------------------------------%
\begin{document}
\flushbottom % Makes all text pages the same height
\maketitle % Print the title and abstract box
\renewcommand{\contentsname}{Conteúdo}
\tableofcontents % Print the contents section
\thispagestyle{empty} % Removes page numbering from the first page

%-----------------------------------------------------------------------------%
%	ARTICLE CONTENTS
%-----------------------------------------------------------------------------%
\section*{Método de Análise: Titulação } % The \section*{} command stops section numbering
\addcontentsline{toc}{section}{Método de Análise: Titulação} % Adds this section to the table of contents
O método de análise apresentado aqui baseia-se no descrito por Winkler modificado por Strickland e Parsons (1972).
%-----------------------------------------------------------------------------%
\section{Fundamento Analítico}

A determinação de \ce{O2} dissolvido em águas naturais é feita pela técnica desenvolvida por Winkler em 1888. Algumas modificações foram introduzidas no decorrer dos anos para facilitar o trabalho a bordo e minimizar algumas fontes de erros. A amostra coletada é rapidamente fixada com cloreto ou sulfato de manganês e hidróxido de potássio alcalino (KI + KOH). O \ce{Mn(OH)2} reage com o oxigênio presente na amostra formando um composto de manganês trivalente. Assim, o \ce{O2} dissolvido é fixado.

Na análise, após acidificação, o composto trivalente de manganês é dissolvido, oxidando \ce{I-} a \ce{I2}, que é estabilizado pela formação de \ce{I3-} devido ao excesso de \ce{I-}. A quantidade de \ce{I3-} formado é proporcional à quantidade de \ce{O2} da amostra e é determinado pela titulação com solução padronizada de tiossulfato de sódio (\ce{Na2S2O3}).

Uma solução de amido, que forma um complexo azul com \ce{I2}, é utilizada para indicar o ponto de equivalência desta titulação.

As equações estequiométricas envolvidas nesta determinação são:

\textit {\underline{ Etapa de Fixação (imediatamente após a coleta)}}
\vspace{0.25cm}

\ce { Mn2+ + 2OH- -> Mn(OH)2_{(s)} v} (branco)
\vspace{0.15cm}

\ce {2Mn(OH)2 + 1/2 O2 + H2O -> 2Mn(OH)3 v} (marrom)
\vspace{0.15cm}

\textit{\underline {Etapa de acidificação (antes da titulação)}}
\vspace{0.25cm}

\ce {2Mn(OH)3 + 6H+ + 2I- -> 2Mn2+ + I2 +6H2O}
\vspace{0.15cm}

\ce {I2 + I- -> I3-}
\vspace{0.15cm}

O iodo molecular na presença de excess de iodeto forma o complexo triiodeto \ce{I3-}
\vspace{0.15cm}

\textit{\underline {Etapa Titulação}}
\vspace{0.25cm}

\ce {I3^{-}_{(aq)} + 2(S2O3)^{2-}_{(aq)} -> 3I^-_{(aq)} + (S4O6)^{2-}_{(aq)}}
\vspace{0.15cm}

O iodo formado na reação é equivalente a quantidade de oxigênio da amostra e é titulado com solução padronizada de tiossulfato de sódio usando-se como indicador o amido.
\vspace{0.25cm}

\textbf{Faixa de Concentração:} de 0,0025 a 4 mM.

%\textbf{Precisão:}o desvio padrão é igual a ±0,01 $\times$ 10$^{-3}$.
%----------------------------------------------------------19,375-------------------%
\section{Amostragem e Estocagem}


As amostras para determinação de oxigênio devem ser coletadas em frascos de vidro de borosilicato de cor âmbar com tampa de viro esmerilhada. 

Quando a amostra for recolhida com garrafa coletora lançada no ambiente, na torneira de saída da água deve ser acoplada uma mangueira de borracha flexível. Essa mangueira deve ser introduzida até o fundo do frasco e mantido nessa posição até o frasco encher, a amostra é deixada transbordar por cerca de um segundo, com o mínimo de agitação e turbulência. 

A amostra para análise de OD deve ser a primeira a ser retirada da garrafa de coleta para evitar-se a expulsão dos gases devido ao aumento da temperatura.

Logo em seguida, adicionar 0,5 mL de cloreto de manganês (II) e 0,5 mL de iodeto alcalino e colocar a tampa sem deixar bolhas de ar. Agitar o frasco vigorosamente para que todo o hidróxido de manganês reaja com o oxigênio da amostra.

Após fixação do oxigênio dissolvido, aguardar toda a precipitação antes de processar a análise durante pelo menos meia hora. As amostras podem ser armazenadas por até 12 horas após a fixação, desde que protegidas da luz e de grandes variações térmicas.

%-----------------------------------------------------------------------------%
\section{Procedimentos Analíticos}

\subsection{Reagentes}

% [noitemsep] removes whitespace between the items for a compact look
\begin{enumerate}[noitemsep]
\item \textbf {Solução de cloreto de manganês II (2 M) - R1} - Dissolver 40 g de \ce{MnCl2.4H2O} ou 32 g de \ce{MnSO4.H2O} com uma pequena quantidade de água destilada e completar o volume em balão volumétrico de 100mL.
\item \textbf{Solução de iodeto alcalino (KI = 3,6M e KOH = 5,4M) - R2} - Dissolver 60 g de KI e 30 g de KOH, separadamente, na menor quantidade de água possível, tendo o cuidado para que a soma das alíquotas não ultrapasse 100mL. Transferir para um balão volumétrico de 100mL e completar o volume com água destilada.
\item \textbf{Ácido sulfúrico} - Adicionar, cuidadosamente, 50 mL de \ce{H2SO4} (95 -- 97\%) a 50mL de água destilada (usar resfriamento durante o processo).
\item \textbf{Solução de tiossulfato de sódio (0,02M)} - Dissolver 5 g de \ce{Na2S2O3} em alíquotas e completar o volume em um balão volumétrico de 1000mL.
\item \textbf{Solução de amido} - Dispersar 1g de amido solúvel em 100mL de água destilada e aquecer a solução, deixando ebulir durante 1 minuto. Esta solução não se conserva por mais de uma semana.
\item \textbf{Solução de iodato de potássio (0,1667x10$^{-2}$ M)} - Dissolver, cuidadosamente 356,7mg de \ce{KIO3}, transferir para um balão volumétrico de 1000 mL e completar com água destilada.
\end{enumerate}

\subsection{Determinação do branco de reagentes}
% [noitemsep] removes whitespace between the items for a compact look
\begin{enumerate}[noitemsep]
	
\item Colocar 50 mL de água destilada em um erlenmeyer de 150mL.
\item Adicionar 1 mL de ácido sulfúrico e agitar.
\item Adicionar 0,5 mL de iodeto alcalino e agitar.
\item Adicionar e 0,5 mL de cloreto de manganês II e adicionar. 
\item Adicionar 1 mL de solução de iodato de potássio (padrão).
\item Titular com solução de tiossulfato de sódio até obter uma coloração ligeiramente amarela. 
\item Adicionar 0,5 mL do indicador de amido 
\item Continuar a titulação até a descoloração. 
\item Anotar o volume gasto de tiossulfato de sódio (V1).
\item Adicionar mais 1 mL de solução de iodato de potássio.
\item Continuar a titulação até a descoloração e anotar o volume gasto de tiossulfato de sódio (V2).
\end{enumerate}


\subsection{Padronização do tiossulfato de sódio}
%[noitemsep] removes whitespace between the items for a compact look
\begin{enumerate}[noitemsep]

\item Colocar 50 mL de água destilada em um erlenmeyer de 150 mL. 
\item Adicionar 1 mL de ácido sulfúrico e agitar.
\item Adicionar 0,5 mL de iodeto alcalino e agitar
\item Adicionar 0,5 mL de cloreto de manganês II e agitar.
\item Adicionar 10 mL da solução de iodato padrão com uma pipeta volumétrica.
\item Titular com solução de tiossulfato de sódio até se obter uma coloração ligeiramente amarela. 
\item Adicionar 0,5 mL do indicador de amido e continuar a titulação até a descoloração. 
\item Repetir a padronização três vezes. Anotar os volumes gastos.

\end{enumerate}

\subsection{Titulação das Amostras}
%[noitemsep] removes whitespace between the items for a compact look
\begin{enumerate}[noitemsep]
\item Adicionar 1 mL de ácido sulfúrico no  frasco onde a amostra foi coletada e fixada.
\item Agitar até a completa dissolução do precipitado. 
\item Transferir 50 mL da amostra para um erlenmeyer de 150 mL. 
\item Titular a amostra com a solução de tiossulfato de sódio até a cor amarelo claro. 
\item Adicionar 0,5 mL de amido e continuar a titulação até o ponto de equivalência (azul para incolor).

\end{enumerate}

%-----------------------------------------------------------------------------%
\section{Cálculos}

\indent 
\textbf {Branco}

\begin{center}
$V_{branco} = V_2 - V_1$
\end{center}

\noindent
\textbf {Padronização do Tiossulfato de Sódio}

\begin{center}
$M_{TS} = [(6 \times M_{IP}) \times V_{IP}]/ (V_{TS} - V_{branco}) $
\end {center}
    
Onde:

$M_{TS}$ = concentração da solução de tiossulfato de sódio (mol L$^{-1}$)

$M_{IP}$ = concentração de solução de iodato padrão (mol L$^{-1}$)

$V_{IP}$ = volume da solução de iodato padrão (mL)

$V_{TS}$ = média dos volumes de solução de tiossulfato gastos na titulação (mL)

\vspace{0.15cm}
\noindent
\textbf {Concentração de Oxigênio }

\begin{center}
$[O_2] = [5603,5 \times M_{TS} \times (V_{TS} -  V_{branco})] / V_0 - V_R$
\end{center}

Onde:

$M_{TS}$ = concentração da solução de tiossulfato de sódio (mol.L-1)

$V_{TS}$ = média dos volumes de solução de tiossulfato gastos na titulação da amostra(mL)

$V_0$ =  volume total da amostra

$V_R$ = volume de reagentes adicionados antes da liberação do iodo (1mL)

%-----------------------------------------------------------------------------%
%	Lista de Referencias
%-----------------------------------------------------------------------------%
%\phantomsection
\section{Referências}
Strickland, J. D. H. e Parsons, T. R. (1972). A practical handbook of seawater analysis. $2^o$ Ed. - Ottawa: Fisheries Research Board of Canada, Bulletin 167, 311p.

\bibliographystyle{unsrt}
% \bibliography{sample}


\end{document}



